\documentclass[manuscript, letterpaper]{aastex6}

% to-do list
% ----------

% style notes
% -----------
% - This file generates by Makefile; don't be typing ``pdflatex'' or some bullshit.
% - Line break between sentences to make the git diffs readable.
% - Use \, as a multiply operator.
% - Reserve () for function arguments; use [] or {} for outer shit.
% - Use \sectionname not Section, \figname not Figure, \documentname not Article or Paper or paper.

\include{gitstuff}
\include{aastexmods}

% packages
\definecolor{cbblue}{HTML}{3182bd}
\usepackage{microtype}  % ALWAYS!
\usepackage{amsmath}
\hypersetup{backref,breaklinks,colorlinks,urlcolor=cbblue,linkcolor=cbblue,citecolor=black}

% define macros for text
\newcommand{\project}[1]{\textsl{#1}}
\newcommand{\acronym}[1]{{\small{#1}}}
\newcommand{\gaia}{\project{Gaia}}
\newcommand{\documentname}{\textsl{Article}}
\newcommand{\sectionname}{Section}
\newcommand{\figname}{Figure}

% define macros for math
\newcommand{\given}{\,|\,}
\newcommand{\dd}{\mathrm{d}}
\newcommand{\transp}[1]{{#1}^{\mathsf{T}}}
\newcommand{\inv}[1]{{#1}^{-1}}
\newcommand{\bs}[1]{\boldsymbol{#1}}
\newcommand{\vperp}{\bs{v}^\perp}
\newcommand{\propm}{\bs{\mu}}
\newcommand{\matrx}[1]{\mathbf{#1}}
\newcommand{\kms}{\rm km~s^{-1}}

% TODO
\newcommand{\todo}[1]{{\color{red}TODO: #1}}

\begin{document}\sloppy\sloppypar\raggedbottom\frenchspacing % trust me

\title{Wide binaries in Gaia DR1}
\author{People}

% Affiliations
% \newcommand{\pu}{1}
% \newcommand{\adrn}{2}
% \newcommand{\ccpp}{3}
% \newcommand{\mpia}{4}
% \newcommand{\uw}{5}
% \newcommand{\sagan}{6}

% \altaffiltext{\pu}{Department of Astrophysical Sciences,
%                    Princeton University, Princeton, NJ 08544, USA}
% \altaffiltext{\adrn}{To whom correspondence should be addressed:
%                      \texttt{adrn@princeton.edu}}
% \altaffiltext{\ccpp}{Center for Cosmology and Particle Physics,
%                      Department of Physics,
%                      New York University, 4 Washington Place,
%                      New York, NY 10003, USA}
% \altaffiltext{\mpia}{Max-Planck-Institut f\"ur Astronomie,
%                      K\"onigstuhl 17, D-69117 Heidelberg, Germany}
% \altaffiltext{\uw}{Astronomy Department, University of Washington,
%                    Seattle, WA 98195, USA}
% \altaffiltext{\sagan}{Sagan Fellow}

\begin{abstract}
Blerg.
% Context
% Aims
% Methods
% Results
\end{abstract}

\keywords{
  methods: data analysis
  ---
  methods: statistical
}

\section{Introduction} \label{sec:intro}

\section{Methods} \label{sec:methods}

We want to evaluate the likelihood ratio
\begin{equation}
  \frac{p(\propm_1, \propm_2)_{\vperp_1 = \vperp_2}}
  {p(\propm_1, \propm_2)_{\vperp_1 \neq \vperp_2}}
\end{equation}
where
\begin{multline}
  p(\propm_1, \propm_2)_{\vperp_1 = \vperp_2} =
    \int \, \dd d_1 \, \dd d_2 \, \dd \vperp \,
    p(\propm_1 \given d_1, \vperp, \matrx{C}_{\mu,1}) \,
    p(\propm_2 \given d_2, \vperp, \matrx{C}_{\mu,2}) \\
    p(d_1 \given \varpi_1, \sigma_{\varpi_1}) \,
    p(d_2 \given \varpi_2, \sigma_{\varpi_2}) \,
    p(\vperp)
\end{multline}
and
\begin{multline}
  p(\propm_1, \propm_2)_{\vperp_1 \neq \vperp_2} =
    \int \, \dd d_1 \, \dd d_2 \, \dd \vperp_1 \, \dd \vperp_2 \,
    p(\propm_1 \given d_1, \vperp_1, \matrx{C}_{\mu,1}) \,
    p(\propm_2 \given d_2, \vperp_2, \matrx{C}_{\mu,2}) \\
    p(d_1 \given \varpi_1, \sigma_{\varpi_1}) \,
    p(d_2 \given \varpi_2, \sigma_{\varpi_2}) \,
    p(\vperp_1) \, p(\vperp_2)
\end{multline}
We'll assume the sky positions $(\alpha, \delta)$ are known perfectly, and that
the uncertainties in proper motion components, $\propm=(\mu_\alpha,
\mu_\delta)$, are Gaussian with known covariance matrix $\matrx{C}_\mu$ so that
\begin{equation}
  p(\propm \given d, \vperp, \matrx{C}) = \left[\det\left(\frac{\matrx{C}_\mu^{-1}}{2\pi}\right)\right]^{1/2} \, \exp \left[ -\frac{1}{2} \transp{\left(\propm - \frac{\vperp}{d}\right)} \, \matrx{C}_\mu^{-1} \, \left(\propm - \frac{\vperp}{d}\right) \right]
\end{equation}

Our distance posterior pdfs, $p(d \given \varpi, \sigma_\varpi)$, will come from
Coryn Bailer-Jones.

Our priors on tangential velocity components, $i$, will be something like:
\begin{equation}
  p(v^\perp_i) = \mathcal{N}(0, 30) \, \kms
\end{equation}

\acknowledgements

This research was partially supported by the \acronym{NSF} (grants
  \acronym{IIS-1124794}, \acronym{AST-1312863}, \acronym{AST-1517237}),
  \acronym{NASA} (grant \acronym{NNX12AI50G}),
  and the Moore-Sloan Data Science Environment at \acronym{NYU}. The data
analysis presented in this article was partially performed on computational
resources supported by the Princeton Institute for Computational Science and
Engineering (PICSciE) and the Office of Information Technology's High
Performance Computing Center and Visualization Laboratory at Princeton
University.

\software{The code used in this project is available from
\url{https://github.com/smoh/gaia-wide-binaries} under the MIT open-source
software license. This version was generated at git commit
\texttt{\githash\,(\gitdate)}.
This research additionally utilized:
    \texttt{Astropy} (\citealt{Astropy-Collaboration:2013}),
    %\texttt{emcee} (\citealt{Foreman-Mackey:2013}),
    \texttt{IPython} (\citealt{Perez:2007}),
    \texttt{matplotlib} (\citealt{Hunter:2007}),
    and \texttt{numpy} (\citealt{Van-der-Walt:2011}).}

% \facility{\sdssiii, \apogee}

\bibliographystyle{aasjournal}
\bibliography{refs}

\end{document}
