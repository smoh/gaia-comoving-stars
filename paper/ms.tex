\documentclass[manuscript, letterpaper]{aastex6}

% to-do list
% ----------

% style notes
% -----------
% - This file generates by Makefile; don't be typing ``pdflatex'' or some bullshit.
% - Line break between sentences to make the git diffs readable.
% - Use \, as a multiply operator.
% - Reserve () for function arguments; use [] or {} for outer shit.
% - Use \sectionname not Section, \figname not Figure, \documentname not Article or Paper or paper.

\include{gitstuff}
\include{aastexmods}

% packages
\definecolor{cbblue}{HTML}{3182bd}
\usepackage{microtype}  % ALWAYS!
\usepackage{amsmath}
\hypersetup{backref,breaklinks,colorlinks,urlcolor=cbblue,linkcolor=cbblue,citecolor=black}

% define macros for text
\newcommand{\project}[1]{\textsl{#1}}
\newcommand{\acronym}[1]{{\small{#1}}}
\newcommand{\gaia}{\project{Gaia}}
\newcommand{\rave}{\project{\acronym{RAVE}}}
\newcommand{\apogee}{\project{\acronym{APOGEE}}}
\newcommand{\documentname}{\textsl{Article}}
\newcommand{\sectionname}{Section}
\newcommand{\figname}{Figure}

% define macros for math
\newcommand{\given}{\,|\,}
\newcommand{\dd}{\mathrm{d}}
\newcommand{\transp}[1]{{#1}^{\mathsf{T}}}
\newcommand{\inv}[1]{{#1}^{-1}}
\newcommand{\bs}[1]{\boldsymbol{#1}}
\newcommand{\vperp}{\bs{v}^\perp}
\newcommand{\propm}{\bs{\mu}}
\newcommand{\matrx}[1]{\mathbf{#1}}
\newcommand{\kms}{\rm km~s^{-1}}

% TODO
\newcommand{\todo}[1]{{\color{red}TODO: #1}}

\begin{document}\sloppy\sloppypar\raggedbottom\frenchspacing % trust me

\title{Wide binaries in Gaia DR1}
\author{People}

% Affiliations
% \newcommand{\pu}{1}
% \newcommand{\adrn}{2}
% \newcommand{\ccpp}{3}
% \newcommand{\mpia}{4}
% \newcommand{\uw}{5}
% \newcommand{\sagan}{6}

% \altaffiltext{\pu}{Department of Astrophysical Sciences,
%                    Princeton University, Princeton, NJ 08544, USA}
% \altaffiltext{\adrn}{To whom correspondence should be addressed:
%                      \texttt{adrn@princeton.edu}}
% \altaffiltext{\ccpp}{Center for Cosmology and Particle Physics,
%                      Department of Physics,
%                      New York University, 4 Washington Place,
%                      New York, NY 10003, USA}
% \altaffiltext{\mpia}{Max-Planck-Institut f\"ur Astronomie,
%                      K\"onigstuhl 17, D-69117 Heidelberg, Germany}
% \altaffiltext{\uw}{Astronomy Department, University of Washington,
%                    Seattle, WA 98195, USA}
% \altaffiltext{\sagan}{Sagan Fellow}

\begin{abstract}
Blerg.
% Context
% Aims
% Methods
% Results
\end{abstract}

\keywords{
  methods: data analysis
  ---
  methods: statistical
}

\section{Introduction} \label{sec:intro}

\section{Methods} \label{sec:methods}

For a pair of stars $(i,j)$, we want to ratio the fully marginalized likelihood
(FML) of the stars having the same 3-space velocity, $\bs{v}=(v^\perp_\alpha,
v^\perp_\delta, v_r)$, compared to the FML of the stars having different 3-space
velocities:
\begin{equation}
  \frac{p(\propm_i, \varpi_i, \matrx{C}_i, \propm_j, \varpi_j, \matrx{C}_j)_{\bs{v}_i = \bs{v}_j}}
  {p(\propm_i, \varpi_i, \matrx{C}_i, \propm_j, \varpi_j, \matrx{C}_j)_{\bs{v}_i \neq \bs{v}_j}}.
\end{equation}
When this ratio is large---that is, the likelihood that the stars have the same
3-space velocity is large and the the likelihood that the stars have different
3-space velicities is small---the pair is a candidate binary star. For \gaia\
DR1, we only observe a parallax, $\varpi$, and two proper motion components,
$\propm = (\mu_\alpha, \mu_\delta)$. Where there is overlap with \rave\ or
\apogee, some number of observed radial velocities, $\tilde{v}_r$, with observed
variances $\sigma^2_{v_r}$ (independent of the \gaia\ data) may also be
available. The likelihood expressions above are marginalized over true 3-space
velocity and distance for each star in the pair. For the numerator,
\begin{equation}
  p(\propm_i, \varpi_i, \matrx{C}_i, \propm_j, \varpi_j, \matrx{C}_j)_{\bs{v}_i = \bs{v}_j} =
    \int \, \dd \bs{v} \, \dd d_i \, \dd d_j \,
    p(\propm_i, \varpi_i, \bs{v}, d_i, \matrx{C}_i) \,
    p(\propm_j, \varpi_j, \bs{v}, d_j, \matrx{C}_j)
\end{equation}
where
\begin{align}
  p(\propm, \varpi, \bs{v}, d, \matrx{C}) &=
    p(\propm, \varpi \given \vperp, d, \matrx{C}) \, p(\vperp, v_r) \, p(\tilde{v}_r \given v_r, \sigma^2_{v_r}) \, p(d \given l,b) \\
  p(\propm, \varpi \given \vperp, d, \matrx{C}) &=
    \left[\det\left(\frac{\matrx{C}^{-1}}{2\pi}\right)\right]^{1/2} \,
    \exp \left[ -\frac{1}{2} \transp{\left(\bs{x} - \bs{x}_\theta \right)} \,
    \matrx{C}^{-1} \,
    \left(\bs{x} - \bs{x}_\theta \right) \right] \\
  \bs{x} &= \transp{\left(\varpi, \mu_\alpha, \mu_\delta\right)} \\
  \bs{x}_\theta &= \transp{\left(\frac{1}{d}, \frac{v^\perp_\alpha}{d}, \frac{v^\perp_\delta}{d}\right)}
\end{align}

We assume Gaussian priors on the true velocity components
\begin{align}
  p(\bs{v}) = \mathcal{N}(0,25)~\kms
\end{align}
and take one of the distance priors from Coryn Bailer-Jones,

For the denominator of the likelihood ratio,
\begin{multline}
  p(\propm_i, \varpi_i, \matrx{C}_i, \propm_j, \varpi_j, \matrx{C}_j)_{\bs{v}_i \neq \bs{v}_j} = \\
    \int \, \dd \bs{v}_i \, \dd \bs{v}_j \, \dd d_i \, \dd d_j \,
    p(\propm_i, \varpi_i, \bs{v}_i, d_i, \matrx{C}_i) \,
    p(\propm_j, \varpi_j, \bs{v}_j, d_j, \matrx{C}_j)
\end{multline}


\acknowledgements

This research was partially supported by the \acronym{NSF} (grants
  \acronym{IIS-1124794}, \acronym{AST-1312863}, \acronym{AST-1517237}),
  \acronym{NASA} (grant \acronym{NNX12AI50G}),
  and the Moore-Sloan Data Science Environment at \acronym{NYU}. The data
analysis presented in this article was partially performed on computational
resources supported by the Princeton Institute for Computational Science and
Engineering (PICSciE) and the Office of Information Technology's High
Performance Computing Center and Visualization Laboratory at Princeton
University.

\software{The code used in this project is available from
\url{https://github.com/smoh/gaia-wide-binaries} under the MIT open-source
software license. This version was generated at git commit
\texttt{\githash\,(\gitdate)}.
This research additionally utilized:
    \texttt{Astropy} (\citealt{Astropy-Collaboration:2013}),
    %\texttt{emcee} (\citealt{Foreman-Mackey:2013}),
    \texttt{IPython} (\citealt{Perez:2007}),
    \texttt{matplotlib} (\citealt{Hunter:2007}),
    and \texttt{numpy} (\citealt{Van-der-Walt:2011}).}

% \facility{\sdssiii, \apogee}

\bibliographystyle{aasjournal}
\bibliography{refs}

\end{document}
