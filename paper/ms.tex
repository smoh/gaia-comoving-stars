\documentclass[manuscript, letterpaper]{aastex6}

% to-do list
% ----------
% - check for TODO's and references to APW, HOGG, SMOH
% style notes
% -----------
% - This file generates by Makefile; don't be typing ``pdflatex'' or some bullshit.
% - Line break between sentences to make the git diffs readable.
% - Use \, as a multiply operator.
% - Reserve () for function arguments; use [] or {} for outer shit.
% - Use \sectionname not Section, \figname not Figure, \documentname not Article or Paper or paper.

\include{gitstuff}
\include{aastexmods}

% packages
\definecolor{cbblue}{HTML}{3182bd}
\usepackage{microtype}  % ALWAYS!
\usepackage{amsmath}
\hypersetup{backref,breaklinks,colorlinks,urlcolor=cbblue,linkcolor=cbblue,citecolor=black}

% define macros for text
\newcommand{\project}[1]{\textsl{#1}}
\newcommand{\acronym}[1]{{\small{#1}}}
\newcommand{\gaia}{\project{Gaia}}
\newcommand{\rave}{\project{\acronym{RAVE}}}
\newcommand{\apogee}{\project{\acronym{APOGEE}}}
\newcommand{\documentname}{\textsl{Article}}
\newcommand{\sectionname}{Section}
\newcommand{\figname}{Figure}
\newcommand{\eqname}{Equation}
\newcommand{\dr}{\acronym{DR1}}
\newcommand{\tgas}{\acronym{TGAS}}

% define macros for math
\newcommand{\given}{\,|\,}
\newcommand{\normal}{{\mathcal{N}}}
\newcommand{\dd}{\mathrm{d}}
\newcommand{\transp}[1]{{#1}^{\!\mathsf{T}}}
\newcommand{\inv}[1]{{#1}^{-1}}
\newcommand{\bs}[1]{\boldsymbol{#1}}
\newcommand{\vperp}{\bs{v}^\perp}
\newcommand{\propm}{\bs{\mu}}
\newcommand{\mat}[1]{\mathbf{#1}}
\renewcommand{\vec}[1]{\bs{#1}}
\newcommand{\kms}{\ensuremath{\rm km~s^{-1}}}
\newcommand{\msun}{{\rm M}_\odot}
\newcommand{\data}{\mathrm{data}}
\newcommand{\snr}{[S/N]_\varpi}
\newcommand{\eye}{\mathbb{I}}

% TODO
\newcommand{\todo}[1]{{\color{red}TODO: #1}}

\begin{document}\sloppy\sloppypar\raggedbottom\frenchspacing % trust me

\title{Co-moving stars in \textsl{Gaia DR1}}
\author{Semyeong Oh\altaffilmark{\pu,\lead},
        Adrian M. Price-Whelan\altaffilmark{\pu},
        David W. Hogg\altaffilmark{\ccpp,\mpia},
        Timothy D. Morton\altaffilmark{\pu},
        David N. Spergel\altaffilmark{\pu,\cca}
}

% Affiliations
\newcommand{\pu}{1}
\newcommand{\lead}{2}
\newcommand{\ccpp}{3}
\newcommand{\mpia}{4}
\newcommand{\cca}{5}

\altaffiltext{\pu}{Department of Astrophysical Sciences,
                   Princeton University, Princeton, NJ 08544, USA}
\altaffiltext{\lead}{To whom correspondence should be addressed:
                     \texttt{semyeong@astro.princeton.edu}}
\altaffiltext{\ccpp}{Center for Cosmology and Particle Physics,
                     Department of Physics,
                     New York University, 4 Washington Place,
                     New York, NY 10003, USA}
\altaffiltext{\mpia}{Max-Planck-Institut f\"ur Astronomie,
                     K\"onigstuhl 17, D-69117 Heidelberg, Germany}
\altaffiltext{\cca}{Simons Center for Computational Astrophysics, ...,
                    New York, NY XXXXX, USA}

\begin{abstract}
% Context
The primary sample of the \gaia\ \textsl{First Data Release} is the brand-new
\textsl{Tycho-Gaia Astrometric Solution}. The precision and size of this sample
creates an opportunity to find new binary stars and moving groups, especially
rare binaries or sparsely populated groups.
% Aims
Here we seize this opportunity.
% Methods
We use a justified marginalized likelihood ratio test to separate
pairs of stars with surprisingly similar three-space velocities from
those consistent with being drawn independently from the field
population.  Although we perform some visualizations using a
(bias-corrected) inverse parallax as a distance, the likelihood test
works in the observable space and uses the \gaia\ noise model
responsibly.
% Results
We find pairs of co-moving stars out to very wide
separations, including separations of a few parsecs! There does not
seem to be any tidal-disruption feature in the separation distribution
at sub-parsec scales. Pairs at separations beyond a
parsec---whether they are bound in a multiple system or drifting
apart---must be very short-lived. The photometric properties of the
members of these pairs are consistent with youth. The prospects for
testing stellar models and chemical abundance measurements are
superficially discussed.
\end{abstract}

\keywords{
  binaries: visual
  ---
  methods: statistical
  ---
  open clusters and associations: general
  ---
  parallaxes
  ---
  proper motions
  ---
  stars: formation
}

\section{Introduction} \label{sec:intro}

Two stars moving with similar full-space velocity vectors (``co-moving stars'')
are either members of a widely-separated stellar multiplet, or remnants of a
dispersing stellar system.

Widely-separated binary stars are weakly bound binary star systems with
semi-major axes larger than $a \gtrsim 10^{-3}~{\rm pc}$. It is generally
assumed that the stars in a wide binary form from gas with the same chemical
composition and are separated enough that there is no interaction or minimal
cross-pollution as the more massive star evolves (\citealt{todo}). Because the
binding energy of these systems is so low, if wide binary stars survive for many
Gigayears their existence alone places limits on the tidal field of the Galaxy
(\citealt{todo}) and the graininess of the gravitational field (\citealt{todo}).
In this picture, a remaining puzzle is how these systems survive the dense
stellar environments of the birth clusters in which they form. Wide binaries are
therefore important tools for studying small-scale star formation and the
large-scale Galactic mass distribution. These stars will appear to be co-moving
with velocity differences comparable to the binary orbital velocity at the
given separation.

At the largest separations, $a \gtrsim 1~{\rm pc}$, co-moving stars are likely
either remnants of disrupting stellar systems or young wide binary stars: at
these extreme separations, the Galactic tidal field would disrupt binary stars
within $t_{\rm disrupt} \lesssim XX~{\rm Gyr}$. Unbound open clusters with
velocity dispersions $\sigma_v \lesssim 1~\kms$ in the Galactic disk disrupt
quickly leaving small streams of stars with the same chemical abundances
(\citealt{chemicaltaggingstuff}). Disrupting or disrupted globular clusters and
dwarf galaxies also produce streams of stars with similar chemical properties
but with larger velocity differences (\citealt{streampeeps}). Stars from such
systems at similar orbital phases will appear to be co-moving with velocity
differences comparable to the internal velocity dispersions, $\sigma_v \approx
1$--$10~\kms$.

To date, thousands of candidate co-moving star pairs have been identified by
searching for stars with common proper motions (\citealt{Luyten:1979,
Poveda:1994, Allen:2000, Gould:2003, Chaname:2004, Lepine:2007,
Alonso-Floriano:2015}). \todo{how is our search different? (1) we have distance, (2) bigger volume than Hipparcos, (3) we use the data properly, (4) we don't cut on distance or on magnitude of proper motion -- we cut on parallax S/N}
% Cite:
% Tokovinin & Lepine 2012
% https://ui.adsabs.harvard.edu/#abs/2012ApJ...757..170A/abstract
% https://ui.adsabs.harvard.edu/#abs/2012AJ....144..102T/abstract
% https://ui.adsabs.harvard.edu/#abs/2014ApJ...790..158A/abstract

% Many searches find things close in angular separation, but need pms / distance

% Two stars moving with full-space velocity vector differences $\lesssim 1~\kms$
% with orbital semi-major axes $a \gtrsim 10^-3~{\rm pc}$.

Useful for:
- star cluster disruption and dispersal
- chemical abundance comparison (no interaction)
- dynamics

In either case, co-moving stars are of great interest
for studying the destruction of star clusters and dwarf galaxies, the tidal
field and small-scale properties of the mass distribution in the Milky Way, and
chemical uniformity and star formation processes.

\section{Data} \label{sec:data}

The primary data set used in this \documentname\ is the Tycho-Gaia Astrometric
Solution (\tgas), released as a part of Data Release 1 (\dr) of the Gaia mission
\citep{2016arXiv160904172G,2016arXiv160904303L}.
\tgas\ contains astrometric measurements (sky position,
parallax, and proper motions) and associated covariance matrices for a large
fraction of the \project{Tycho-2} catalog \citep{2000A&A...355L..27H} with median
astrometric precision comparable to that of the \project{Hipparcos} catalog
\citep[$\approx 0.3~{\rm mas}$;][]{2007ASSL..350.....V}. In terms of parallax
signal-to-noise ($\snr = \varpi/\sigma_\varpi$), the \tgas\ catalog contains
42385 high-precision stars with $\snr > 32$.

We construct a sample of candidate pair list as follows.
We first apply a global parallax S/N cut of 8 to \tgas, which leaves 619,618 stars.
Then, for each star, we search a potential co-moving partner star with a
difference in tangential velocity $\Delta v_t < 10$~\kms\ within 10 pc
using RA, DEC, a point-estimate of distance, and the proper motions.
Throughout this \documentname\, we use the following estimator for
the point-estimate of distance applying a correction for the Lutz-Kelker bias
(\citealt{lutzkelker}):
\begin{equation}
  d = 1000 \, \left[\frac{\varpi}{2} \,
    \left(1 + \sqrt{1 - \frac{16}{\snr^2}} \right) \right]^{-1} \, {\rm pc}
\end{equation}
where $\varpi$ is the parallax in mas.
The difference in tangential velocity between two stars is, then,
\begin{equation}
  |\Delta \vec{v}_{\rm t}| = |d_1 \vec\mu_1 - d_2 \vec\mu_2|
\end{equation}
where $\vec\mu = (\mu_\alpha, \mu_\delta)$.

Figure~\ref{fig:dv-sep} shows $|\Delta v_t|$ against the physical separation
for the resulting 271,232 unique candidate pairs.
A few key observations can be made:

\begin{itemize}
  \item At small separation ($<1$~pc), there is a population of pairs with
  very small tangential velocity difference ($<2$~km/s). Given that these pairs
  are very close in both positions and proper motions,
  it is highly probable that they are actually co-moving e.g., as wide binaries.
  
  \item  However, it is not only (wide) binaries which are gravitationally
  bound that are expected to appear as co-moving.
  OB associations, moving groups, and open clusters are also of consideration.
  Since the basic unit of our choice is a pair of stars, we can expect that these astrophysical objects may be detected as networks of pairs.
  As the pair separation increases, the nature of co-moving pairs
  will change from binaries to those related to these larger objects,
  which will tend to subtend a larger angle in sky.
  Under the assumption of having the same 3D velocity,
  two stars of a co-moving pair can essentially be thought of as projections
  of this velocity onto sphere at different viewing angles.
  The larger the difference in viewing angles is, the larger the difference in tangential 
  velocities will be.
  Due to this projection effect, a population of genuine co-moving pairs
  will extend to larger $\Delta v_t$ at larger separation.
  This indeed can be seen in Figure~\ref{fig:dv-sep} as an over-density
  in lower right corner that gets thinner as $\Delta v_t$ increases.
  
  \item Finally, there is a population of “random” pairs of field stars
  which are not actually co-moving, but still has $\Delta v_t < 10$~\kms\
  by chance.
  As $\Delta v_t$ increases, this population will dominate.
  However, it is clear from this Figure that there is an overlap between
  genuine co-moving pairs and “random” pairs.
\end{itemize}

In the following section, we construct a statistical model that propagates
the non-trivial uncertainties in the data to our beliefs about the likelihood
that a given pair of stars is co-moving.

\begin{figure*}[p]
  \begin{center}
    \includegraphics[width=\textwidth]{figures/sep_dvtan.pdf}
  \end{center}
  \caption{%
    Difference in tangential velocity vs. physical separation for candidate
    pairs of stars with separation $< 10$~pc and $|\Delta v_t| < 10$~\kms.
    % Note the points with small velocity difference: these are likely
    % widely-separated binary stars.
    \label{fig:dv-sep}}
\end{figure*}

\section{Methods} \label{sec:methods}

The abundance of pairs of stars with small velocity difference in
\figname~\ref{fig:dv-sep} suggests that there are a
significant number of co-moving stars in the \tgas\ data at a range
of separations.
Here we develop a method to select high-confidence co-moving
stars that properly incorporates the complex uncertainties associated with the
\gaia\ data. We make the following assumptions in order to construct a
statistical model (a likelihood function with explicit priors on our
parameters):
\begin{itemize}
  \item We assume that the uncertainties in the data---parallax, $\varpi$, and
    two proper motion components, $\propm = (\begin{array}[t]{c c} \mu_\alpha &
    \mu_\delta\end{array})^\mathsf{T}$---are Gaussian with known covariances
    $\mat{C}$.
    \todo{SMOH: mention GAIA paper that says Gaussian is good approximation?}
  \item We assume that the 3-space velocities of a given pair of stars
    $(\vec{v}_i, \vec{v}_j)$ in the \tgas\ sample (relative to the solar system
    barycenter) are either (1) co-moving with velocity drawn from a velocity
    prior $p(\vec{v})$ and  velocity difference drawn from a zero-mean Gaussian
    with velocity dispersion $s$, or (2) individually drawn from a velocity
    prior $p(\vec{v})$.
\end{itemize}

Under these assumptions, the likelihood of a proper motion measurement for a
star given true distance, $d$, and true tangential velocity, $\vec{v}^\perp =
(\begin{array}[t]{c c} v_\alpha & v_\delta\end{array})^\mathsf{T}$ is
\begin{align}
  L(\vec{v}, d, s^2) &=
    \left[\det\left(\frac{\tilde{\mat{C}}^{-1}}{2\pi}\right)\right]^{1/2} \,
    \exp \left[ -\frac{1}{2} \transp{\left(\vec{\mu} - \vec{x}_\theta \right)} \,
    \tilde{\mat{C}}^{-1} \,
    \left(\vec{\mu} - \vec{x}_\theta \right) \right] \label{eq:likefn} \\
  \vec{x}_\theta &= d^{-1} \, \vec{v}^\perp
\end{align}
where the tangential velocity $\vec{v}^\perp$ is related to the 3-space velocity
$\vec{v}$ through projection onto a tangent plane at a given star's sky position
$(\alpha, \delta)$
\begin{align}
  \vec{v}^\perp &= \mat{M}\,\vec{v} \\
  & = \left(
      \begin{array}{c c c}
        -\sin\alpha & \cos\alpha & 0 \\
        -\sin\delta \, \cos\alpha & -\sin\delta \, \sin\alpha & \cos\delta
      \end{array}
    \right) \,
    \left(\begin{array}{c} v_x \\ v_y \\ v_z \end{array}\right) \label{eq:transformation}
\end{align}
and the modified covariance matrix $\tilde{\mat{C}}$ is
\begin{equation}
  \tilde{\mat{C}} = \mat{C} + s^2 \, \eye
\end{equation}

We compute the fully marginalized likelihood (FML) that a given pair of stars
has the same 3-space velocity with a small velocity difference drawn from a
zero-mean Gaussian with variance $s^2$ (hypothesis 1, $\mathcal{L}_1$), and
the FML of the stars having different 3-space velocities (hypothesis 2,
$\mathcal{L}_2$).
We use the FML ratio $\mathcal{L}_1/\mathcal{L}_2$ as a scalar for selecting
candidate co-moving stars, as described below in more detail.
To compute these FMLs, the likelihood functions $L_i, L_j$ are marginalized
over true 3-space velocity and distance for each star in the pair $(i,j)$.
\begin{align}
  \mathcal{L}_1 &=
    \int \, \dd d_i \, \dd d_j \, \dd^3 \vec{v} \,
    L_i(\vec{v}, d_i, s^2) \,
    L_j(\vec{v}, d_j, s^2) \,
    p(\vec{v}) \, p(d_i \given \varpi_i) \, p(d_j \given \varpi_j) \\
  \mathcal{L}_2 &=
    \int \, \dd d_i \, \dd d_j \, \dd^3 \vec{v}_i \, \dd^3 \vec{v}_j \,
    L_i(\vec{v}_i, d_i, 0) \,
    L_j(\vec{v}_j, d_j, 0) \,
    p(\vec{v}_i) \, p(\vec{v}_j) \, p(d_i \given \varpi_i) \, p(d_j \given \varpi_j). \label{eq:hyp2}
\end{align}
The marginalization integral for hypothesis 2 can be split into the product of
two simpler integrals $\mathcal{L}_2 = Q_i \, Q_j$ where
\begin{equation}
  Q = \int \, \dd d \, \dd^3 \vec{v} \, L(\vec{v}, d, 0) \, p(\vec{v}) \, p(d\given\varpi)
\end{equation}
If the velocity prior $p(\vec{v})$ is also Gaussian, the integrals over velocity
in both cases can be performed analytically:
\todo{SMOH: is this what we do?}
We use an equal-weight mixture of three isotropic, zero-mean Gaussian
distributions
\begin{equation}
  p(\vec{v}) = \frac{1}{3} \, \sum_{m=1}^3 \, \mathcal{N}(0, \sigma_{v,m}^2)
\end{equation}
with velocity dispersions $(\sigma_{v,1}, \sigma_{v,2}, \sigma_{v,3}) = (8, 32,
128)~\kms$ meant to represent young disk stars, old disk stars, and halo stars.
We derive the relevant expressions in Appendix~\ref{sec:appendix}.
After marginalizing over velocity, the likelihood integrands only depend on
distance; we numerically compute the integrals over the true distances to each
star in a pair, $d_1,d_2$, using Monte Carlo integration with $K$ samples from
the distance posterior pdfs:
\begin{equation}
  \int \, \dd d \, \tilde{L}(d) \, p(d\given\varpi) \approx
    \frac{1}{K} \, \sum_k^K \, \tilde{L}(d_k)
\end{equation}
where $\tilde{L}(d)$ is the velocity-marginalized likelihood function.
Through experimentation, we have found that $K=128$ samples are sufficient for
estimating the above integrals for stars with a wide range in parallax
signal-to-noise.

\todo{APW/SMOH: in results, specify what we set $s^2$ to:}
To search for wide binaries, we set $s^2 = \frac{2 \, G \, \msun}{|\vec{x}_i-\vec{x}_j|}$.
To search for dissolving open clusters, we set $s^2 = 1~(\kms)^2$.


\section{Results}

This section is divided into three parts. First, we discuss and justify a cut of
the likelihood ratio to select candidate co-moving pairs.
Second, we present the statistics and properties of co-moving pairs, and discuss its nature.
Finally, we describe our catalog.

\subsection{Selecting candidate co-moving pairs}

In this section, we examine the distribution of likelihood ratios
$\ln \mathcal{L}_1 /\mathcal{L}_2$, and come up with a reasonable cut
for this quantities to select co-moving pairs from candidate pairs list.




With these remarks in mind, we discuss the likelihood ratio distribution.


Figure~\ref{fig:likelihoodratios} shows the likelihood ratios for all 2.6 million
candidate pairs.
Because we do not carry over the normalizations of all probabilities involved
in $L_1$ and $L_2$, the absolute values of the likelihood ratio is not
a meaningful quantity.
Given our crude way of coming up with candidate pair list, we expect
a huge number of pairs that are not genuinely co-moving, compared to
the smaller number of pairs that are.
In order to gauge an appropriate cut on the likelihood ratios,
we compute the likelihood ratios for 200,000 random pairs of stars with the same
signal-to-noise ratio cut as candidate pairs.
Additionally, we select from candidate pair list, those with very small
difference in tangential velocities ($\Delta v_t < 0.5$~km/s) in
two minimum signal-to-noise ratio\footnote{lower of the two stars in a pair}
bins (S/N$>32$ and S/N$<32$) that are highly probable
to be genuine co-moving pairs.
As expected, we see that the likelihood ratio $\ln \mathcal{L}_1 /\mathcal{L}_2$
distribution peaks at higher values, favoring hypothesis 1, for highly probable
co-moving pairs than the random pairs.
We also see that the values are even higher for higher minimum signal-to-noise ratio pairs.

Based on the comparison with random pairs, we can come up with a cut
on the likelihood ratios that reduces the cumulative contamination due to random pairs
to a given threshold.
The value of the cut, the resulting number of pairs, and
its fraction of the total number of candidate pairs
depending on the contamination rate are listed in
Table~\ref{tab:candidatecounts}.
Because the likelihood ratio $\ln \mathcal{L}_1 /\mathcal{L}_2$ asymtotically
is chi-square distribution, the power-law fall off makes the number of resulting
pairs sensitive to the cut.

\subsection{Statistics, properties, and nature of co-moving pairs}

We present the statistics of the resulting co-moving pairs.
We also match the pairs with known open clusters in the Milky Way to
provide information on cluster membership.

Once we have decided the co-moving pairs from candidate pair list,
we can count how many co-moving neighbors a star has.
This is presented in Figure~\ref{fig:hist_Nneighbors}.
It is clear from this figure that we have many aggregates of co-moving stars, most
likely an open cluster or a moving group. We also see that there is still
a large number of stars which only one or two co-moving neighbors.
In order to select pairs that are not associated with any other pair,
we separate a subset of co-moving pairs whose member is a mutually exclusive
co-moving neighbor to the other member (``mutually exclusive pairs'').
The number of mutually exclusive pairs is listed on the last column of
Table~\ref{tab:candidatecounts}.

\begin{table}[tbh]
  \caption{Number of pairs}
  \label{tab:candidatecounts}
  \centering

  \begin{tabular}{l|c|c|c|r}
  \hline

  \hline
  \textbf{Contamination} & \textbf{$\ln \mathcal{L}_1 /\mathcal{L}_2$ cut} &
    \textbf{Count} & \textbf{Fraction} & \textbf{$N_\textrm{me}$} \\
  \hline
    0.01    \% & 5.33 & 92458 & 0.462 \% & 5055 \\
    0.001   \% & 6.53 & 13354 & 0.067 \% & 1549 \\
    0.0001  \% & 7.29 &  2832 & 0.014 \% & 582  \\
    0.00001 \% & 7.35 &  2495 & 0.012 \% & 532  \\
  \hline
  \end{tabular}
\end{table}


\begin{figure*}[p]
  \begin{center}
    \includegraphics[width=\textwidth]{figures/likelihoodratios.pdf}
  \end{center}
  \caption{%
    Density histogram of likelihood ratios for candidate pairs and random pairs.
    We also show the distribution for highly probable genuine co-moving pairs with
    $\Delta v_t < 0.5$~km/s in high ($S/N>32$) and low ($S/N<32$) signal-to-noise ratio
    bins. In both $S/N$ bins, the highly probable co-moving pairs are clearly
    separated from random pairs while the distribution of $\ln \mathcal{L}_1 /\mathcal{L}_2$
    for candidate pairs is expected to be a combination of
    a huge number of ``random'' pairs that are not actually co-moving, and
    a smaller number of genuine co-moving pairs with varying $S/N$.
    \label{fig:likelihoodratios}}
\end{figure*}

We now examine the separation distribution of co-moving pairs in
Figure~\ref{fig:hist_separation}.

Finally, we show the distribution co-moving pairs in galactic longitude and
distance in Figure~\ref{fig:glon_d_pairlines}.
In the top panel, we show all co-moving pairs with contamination rate of $0.01~\%$
(undersampled by a factor of 4 in order to unclutter the plot) as connected line segments.
We find many aggregates of co-moving pairs, which coincides with the positions of
known Milky Way star clusters \citep{Kharchenko:2016aa} indicated by red circles.
Clumps of co-moving pairs at $l=300-360$~degrees and $d=100-200$~pc correspond
to the location of OB associations
Upper Scorpius, Upper Centaurus Lupus, and Lower Centaurus Crux
\citep{de-Zeeuw:1999aa}.
The strong correlation with known co-moving structures disappears when we only
consider mutually exclusive pairs (bottom panel).

\begin{figure*}[p]
  \begin{center}
    \includegraphics[width=\textwidth]{figures/dist_networksize.pdf}
  \end{center}
  \caption{%
    Histogram of co-moving network sizes.
    \label{fig:hist_Nneighbors}}
\end{figure*}

\begin{figure*}[p]
  \begin{center}
    \includegraphics[width=\textwidth]{figures/hist_separation.pdf}
    \includegraphics[width=\textwidth]{figures/hist_separation_mutexc.pdf}
  \end{center}
  \caption{%
    Top: Separation distribution of co-moving pairs of stars for varying degrees
    of contamination due to random pairs. The distribution peaks around 10~pc,
    which is similar to the typical tidal radius of clusters in the Milky Way
    \citep{Kharchenko:2013aa}.
    Bottom: Same as the previous plot but only for mutually exclusive pairs.
    \label{fig:hist_separation}}
\end{figure*}


\begin{figure*}[p]
  \begin{center}
    \includegraphics[width=\textwidth]{figures/glon_d_pie.pdf}
  \end{center}
  \caption{%
    Panoramic view of co-moving pairs of stars. The angle is the Galactic
    longitude, and the distance is in pc.
    Pair networks of size 2 and 3 are connected by gray and blue lines.
    Pair networks of size 4 and larger are plotted with a unique color.
    Black circles mark the positions of known Milky Way star clusters
    from \citet{Kharchenko:2016aa}.
    \label{fig:glon_d_pairlines}}
\end{figure*}

% \begin{figure*}[p]
%   \begin{center}
%     \includegraphics[width=\textwidth]{figures/glon_d_pairlines.pdf}
%     \includegraphics[width=\textwidth]{figures/glon_d_pairlines_mutexc.pdf}
%   \end{center}
%   \caption{%
%     Top: Distribution of co-moving pairs in galactic longitude and
%     distance. Each co-moving pair is connected by a line.
%     Red circles mark the locations of 135 known Milky Way open clusters within
%     $d<600$~pc from \citealt{Kharchenko:2016aa} while
%     stars in known OB associations \citep{de-Zeeuw:1999aa} are colored
%     by which association they belong.
%     We see that some of the aggregates of co-moving pairs are clearly
%     associated with known open clusters, and OB associations.
%     Bottom: Same as the previous plot but only for mutually exclusive pairs.
%     The mutually exclusive pairs no longer traces any known co-moving structures
%     strongly.
%     \label{fig:glon_d_pie}}
% \end{figure*}

\todo{Galactic context / orbits}

\subsection{Catalog of candidate co-moving pairs}

\section{Conclusions}

\acknowledgements

This research was partially supported by the \acronym{NSF} (grants
  \acronym{IIS-1124794}, \acronym{AST-1312863}, \acronym{AST-1517237}),
  \acronym{NASA} (grant \acronym{NNX12AI50G}),
  and the Moore-Sloan Data Science Environment at \acronym{NYU}. The data
analysis presented in this article was partially performed on computational
resources supported by the Princeton Institute for Computational Science and
Engineering (PICSciE) and the Office of Information Technology's High
Performance Computing Center and Visualization Laboratory at Princeton
University.

\software{The code used in this project is available from
\url{https://github.com/smoh/gaia-wide-binaries} under the MIT open-source
software license. This version was generated at git commit
\texttt{\githash\,(\gitdate)}.
This research additionally utilized:
    \texttt{Astropy} (\citealt{Astropy-Collaboration:2013}),
    %\texttt{emcee} (\citealt{Foreman-Mackey:2013}),
    \texttt{IPython} (\citealt{Perez:2007}),
    \texttt{matplotlib} (\citealt{Hunter:2007}),
    and \texttt{numpy} (\citealt{Van-der-Walt:2011}).}

% \facility{\sdssiii, \apogee}

\bibliographystyle{aasjournal}
\bibliography{refs}

\appendix

\section{Relevant properties of Gaussian integrals}
\label{sec:appendixA}

In what follows, all vectors are column vectors, unless we have transposed them.
A relevant exponential integral solution is
\begin{eqnarray}
  \ln\left[\int\exp(-\frac{1}{2}\,
    \transp{[\vec{x}-\vec{\nu}]} \,
    \inv{\mat{A}} \,
    [\vec{x}-\vec{\nu}] - \Delta) \, \dd \vec{x}\right]
  &=& +\frac{1}{2}\ln ||2\pi\,\mat{A}|| -\Delta
  \quad , \label{eq:gauss-int}
\end{eqnarray}
where $\vec{x}$ and $\vec{\nu}$ are $D$-dimensional vectors, $\mat{A}$ is a
positive definite matrix, $\Delta$ is a scalar, and the integral is over all of
$D$-dimensional $\vec{x}$-space.
To cast our problem in this form, we will need to complete the square of the
exponential argument.
If we equate
\begin{eqnarray}
  \frac{1}{2}\,\transp{[\vec{x}-\vec{\nu}]}\,\inv{\mat{A}}\,[\vec{x}-\vec{\nu}] + \Delta
  &=& \frac{1}{2}\,\transp{\vec{x}}\,\inv{\mat{A}}\,\vec{x} + \transp{\vec{x}}\,\mat{B}\,\vec{b} + C
  \quad ,
\end{eqnarray}
where $\mat{B}\,\vec{b}$ is an $D$-vector, and $C$ is a scalar, then we find
\begin{eqnarray}
  \vec{\nu} &=& -\mat{A}\,\mat{B}\,\vec{b}
  \\
  \Delta & = & C - \frac{1}{2}\,\transp{\vec{\nu}}\,\inv{\mat{A}}\,\vec{\nu}
  \quad .
\end{eqnarray}
We will identify terms in our likelihood functions with $\mat{A}$,
$\mat{B}\,\vec{b}$, and $C$, convert to $\vec{\nu}$ and $\Delta$ and compute
the marginalized likelihood using \eqname~\ref{eq:gauss-int}.

\section{Expressions for the marginalized likelihoods}\label{sec:appendix}

At given distance $d$, the velocity-marginalized likelihood can be computed
analytically using the expressions in Appendix~\ref{sec:appendixA}.
As described above, we use an isotropic, mixture-of-Gaussians prior on velocity
where the velocity dispersion of a given mixture component is $\sigma_{v,m}$ and
the 3-space variance tensor is $\mat{V}_m = \sigma_{v,m}^2 \, \eye$.
The likelihood for the data is also a Gaussian, as shown in
\eqname~\ref{eq:likefn}.
We will start by writing down expressions for the the likelihood multiplied by
the prior pdf for the velocities.
Here we slightly change the notation used in \sectionname~\ref{sec:methods} for
simplicity.

We construct a velocity-space data vector $y$ as follows:
\begin{equation}
  \vec{y} =
    \transp{\left(
      \begin{array}{c c c c}
        d_i\,\mu_{\alpha,i} &
        d_i\,\mu_{\delta,i} &
        d_j\,\mu_{\alpha,j} &
        d_j\,\mu_{\delta,j}
      \end{array}
    \right)}
\end{equation}
where again the subscripts $i,j$ refer to the indices of each star in the pair
and we have multiplied the observables (the proper motions) by the distances
$d_i, d_j$, which is permitted because we are conditioning on the distances.
Fundamentally, our hypothesis 1 model (the stars have the same velocity with a
small difference) is
\begin{equation}
  \vec{y} = \mat{M} \, \vec{v} + \mathrm{noise}
\end{equation}
where now the $4 \times 3$ transformation matrix $\mat{M}$ is a stack of the
transformation matrices for each star computed from the pair of sky positions
and using \eqname~\ref{eq:transformation}.
The noise (in the $y$ vector) is drawn from a $4 \times 4$ Gaussian with
block-diagonal covariance matrix, $\mat{\Sigma}$, constructed from the
covariance matrix of the observables, $\mat{C}_i, \mat{C}_j$, and the distances
$d_i, d_j$:
\begin{equation}
  \mat{\Sigma} = \left(
    \begin{array}{c c}
      d_i^2 \, \mat{C}_i & 0 \\
      0 & d_j^2 \, \mat{C}_j
    \end{array}
  \right)
\end{equation}

Given all these definitions, the likelihood function for hypothesis 1 is
\begin{eqnarray}
  p(\data \given \vec{v}, d_i, d_j) &=& d_i^2\,d_j^2\,
    \normal(\vec{y} \given \mat{M}\,\vec{v}, \mat{\Sigma}) \\
  \ln p(\data \given \vec{v}, d_i, d_j) &=& 2\,\ln d_i + 2\,\ln d_j
    -\frac{1}{2}\,\ln||2\pi\,\mat{\Sigma}|| \nonumber \\
    && \quad -\frac{1}{2}\,\transp{[\vec{y}-\mat{M}\,\vec{v}]}\,
      \inv{\mat{\Sigma}}\,
      [\vec{y}-\mat{M}\,\vec{v}]
  \quad ,
\end{eqnarray}
where the factor of $d_i^2\,d_j^2$ converts to units of one over \gaia\ data
from units of one over $y$, which includes the distances.

Now we multiply this likelihood with the velocity prior (here shown only for one
prior velocity mixture component) and complete the square.
In the original parameterization, we find:
\begin{eqnarray}
  \mat{A} &=& \inv{[\transp{\mat{M}}\,\inv{\mat{\Sigma}}\,\mat{M}+\inv{\mat{V}_m}]}
  \\
  \vec{\nu} &=& -\mat{A}\,\transp{\mat{M}}\,\inv{\mat{\Sigma}}\,\vec{y}
  \\
  \Delta &=& -2\,\ln d_i -2\,\ln d_j
    +\frac{1}{2}\,\ln||2\pi\,\mat{\Sigma}|| +\frac{1}{2}\,\ln||2\pi\,\mat{V}_m|| \nonumber \\
    && \quad +\frac{1}{2}\,\transp{\vec{y}}\,\inv{\mat{\Sigma}}\,\vec{y} -\frac{1}{2}\,\transp{\vec{\nu}}\,\inv{\mat{A}}\,\vec{\nu}
  \quad ,
\end{eqnarray}
which we plug in to \eqname~\ref{eq:gauss-int} to get the marginalized
likelihood conditioned on the two distances $d_i, d_j$.

The marginalized likelihood for the hypothesis 2 model (the stars have
independent velocities) is very similar.
In this case, the marginalized likelihood is a product of two independent
integrals $Q$, composed in the same way as the hypothesis 1 model but now for
each star individually, where
\begin{eqnarray}
  \vec{y} &=&
    \transp{\left(
      \begin{array}{c c}
        d\,\mu_{\alpha} &
        d\,\mu_{\delta}
      \end{array}
    \right)}
  \\
  \mat{\Sigma} &=& d^2 \, \mat{C}
  \\
\end{eqnarray}
and $\mat{M}$ is now the transformation matrix for one star. Then, again,
\begin{eqnarray}
  \mat{A} &=& \inv{[\transp{\mat{M}}\,\inv{\mat{\Sigma}}\,\mat{M}+\inv{\mat{V}_m}]}
  \\
  \vec{\nu} &=& -\mat{A}\,\transp{\mat{M}}\,\inv{\mat{\Sigma}}\,\vec{y}
  \\
  \Delta &=& -2\,\ln d_i -2\,\ln d_j
    +\frac{1}{2}\,\ln||2\pi\,\mat{\Sigma}|| +\frac{1}{2}\,\ln||2\pi\,\mat{V}_m|| \nonumber \\
    && \quad +\frac{1}{2}\,\transp{\vec{y}}\,\inv{\mat{\Sigma}}\,\vec{y} -\frac{1}{2}\,\transp{\vec{\nu}}\,\inv{\mat{A}}\,\vec{\nu}
  \quad ,
\end{eqnarray}
and
\begin{equation}
  Q = \frac{1}{2}\ln ||2\pi\,\mat{A}|| -\Delta \quad .
\end{equation}

\end{document}
